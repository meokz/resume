% LaTeX resume using res.cls
\documentclass[margin]{res}

% \usepackage{style/helvetica} % uses helvetica postscript font (download helvetica.sty)
\usepackage{newcent}   % uses new century schoolbook postscript font


\setlength{\textwidth}{5.1in} % set width of text portion

\begin{document}

\newcommand{\adress}{(adress:) XXXX}
\newcommand{\info}{phone / kaootao@gmail.com / kazukiotao.com}


% Center the name over the entire width of resume:
\moveleft.5\hoffset\centerline{\large\bf Kazki Otao}
% Draw a horizontal line the whole width of resume:
\moveleft\hoffset\vbox{\hrule width\resumewidth height 1pt}\smallskip
% address begins here
% Again, the address lines must be centered over entire width of resume:
\moveleft.5\hoffset\centerline{\adress}
\moveleft.5\hoffset\centerline{Ibaraki, Japan 305-0821}
\moveleft.5\hoffset\centerline{\phone}

\begin{resume}

% \section{OBJECTIVE}
% A position in the field of computers with special
% interests in business applications programming,
% information processing, and management systems.

\section{EDUCATION}

{\sl Master of Science in Informatics} \\
Apr. 2019 to Current (Expected End Date: Mar. 2021) \\
University of Tsukuba, Japan

{\sl Bachelor of Science in Media Sciences and Engineering} \\
Apr. 2017 to Mar. 2019 \\
University of Tsukuba, Japan \\
Digital Nature Group (Adviser: Prof. Yoichi Ochiai) \\
- Researching light field display and retinal projection display using dihedral corner reflector array.

{\sl Associate Degree in Computer Science and Electronic Engineering} \\
Apr. 2012 to Mar. 2017 \\
National Institute of Technology, Tokuyama College, Japan \\
Soft Computing Laboratory (Adviser: Prof. Takanori Koga) \\
- Researching a fog display for visualization of adaptive shape-changing flow.

\section{COMPUTER \\ SKILLS}

{\sl Languages \& Software:}\\
C/C++, C\#, Java, Python, Unity, Visual Basic, Javascript, HTML/CSS,
Git, LaTeX,

{\sl Framework \& Library:}\\
Chainer

{\sl Graphics:}\\
Photoshop, After Effects, Premiere Pro, AviUtl, Blender, Illustrator

\section{PROFESSIONAL \\ EXPERIENCE}

{\sl Creator \\ at IPA (IT Promotion Agency), Japan.} \hfill July. 2019 to Currnet \\

\begin{itemize}
  \item Development of Automatic telop generation for SNS using machine learning.
  \item Grant 2,304,000 Yen: 2019 Exploratory IT Human Resources Project (The MITOH Program).
\end{itemize}

{\sl Reserach Enginner \\ at Pixie Dust Technologies, Inc., Japan.} \hfill Sept. 2017 to Currnet \\

\begin{itemize}  \itemsep -2pt
  \item Optical design and computer graphics processing of head-mounted display, light field display, and aerial display.
\end{itemize}

{\sl Software Enginner \\ at Unirobot Corporation, Japan.} \hfill Dec. 2016 to Aug. 2017 \\

\begin{itemize}  \itemsep -2pt
  \item Development of User interface for home robot "Unibo".
\end{itemize}

\section{PUBLICATION}

{\bf Book} \\
1. Recent Developments and Prospective Applications of Aerial Display. CMC Publishing Co.,Ltd., 2018, 267p. (Written contribution of Part III, Chapter 9 )

{\bf International Conference with Peer Review - Oral Presentation} \\
1. \underline{Kazuki Otao}, Yuta Itoh, Kazuki Takazawa, Hiroyuki Osone, and Yoichi Ochiai. 2018. Air Mounted Eyepiece: Optical See-Through HMD Design with Aerial Optical Functions. In Proceedings of the 9th Augmented Human International Conference (AH ’18). ACM.

2. \underline{Kazuki Otao}, Yuta Itoh, Hiroyuki Osone, Kazuki Takazawa, Shunnosuke Kataoka, and Yoichi Ochiai. 2017. Light field blender: designing optics and rendering methods for see-through and aerial near-eye display. In SIGGRAPH Asia 2017 Technical Briefs (SA '17). ACM.

{\bf International Conference with Peer Review - Posters} \\
1. \underline{Kazuki Otao} and Takanori Koga. 2017. Mistflow: a fog display for visualization of adaptive shape-changing flow. In SIGGRAPH Asia 2017 Posters (SA '17). ACM.

2. Shinnosuke Ando, \underline{Kazuki Otao}, Yoichi Ochiai. 2019. Glass-Beads Display: Evaluation for Aerial Graphics Rendered by Retro-Reflective Particles. In HCI International 2019 Posters (HCII 2019). Springer.

3. Yoichi Ochiai, \underline{Kazuki Otao}, Yuta Itoh, Shouki Imai, Kazuki Takazawa, Hiroyuki Osone, Atsushi Mori, and Ippei Suzuki. 2018. Make your own Retinal Projector: Retinal Near-Eye Displays via Metamaterials. In SIGGRAPH ’18 Posters (SIGGRAPH ’18). ACM.

4. Shinnosuke Ando, \underline{Kazuki Otao}, Kazuki Takazawa, Yusuke Tanemura, and Yoichi Ochiai. 2017. Aerial image on retroreflective particles. In SIGGRAPH Asia 2017 Posters (SA '17). ACM,.

{\bf International Conference with Peer Review - Demos} \\
1. \underline{Kazuki Otao}, Yuta Itoh, Kazuki Takazawa, Hiroyuki Osone, and Yoichi Ochiai. 2018. Transmissive Mirror Device based Near-Eye Displays with Wide Field of View. In SIGGRAPH ’18 Emerging Technologies (SIGGRAPH ’18). ACM.

2. Yoichi Ochiai, \underline{Kazuki Otao}, Yuta Itoh, Shouki Imai, Kazuki Takazawa, Hiroyuki Osone, Atsushi Mori, and Ippei Suzuki. 2018. Make your own Retinal Projector: Retinal Near-Eye Displays via Metamaterials. In SIGGRAPH ’18 Emerging Technologies (SIGGRAPH ’18). ACM.

3. Takahito Aoto, Yuta Itoh, \underline{Kazuki Otao}, Kazuki Takazawa, and Yoichi Ochiai. 2018. A design for optical cloaking display. In SIGGRAPH ’19 Emerging Technologies (SIGGRAPH ’19). ACM.

{\bf International Conference - Invited Talk} \\
Takanori Koga and \underline{Kazuki Otao}. 2018. An Interactive Fog Display to Express Adaptive Shape-Changing Flow. In the 25th International Display Workshops (IDW '18).

\end{resume}
\end{document}




