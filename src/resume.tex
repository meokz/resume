% LaTeX resume using res.cls
\documentclass[margin]{res}

% \usepackage{style/helvetica} % uses helvetica postscript font (download helvetica.sty)
\usepackage{newcent}   % uses new century schoolbook postscript font


\setlength{\textwidth}{5.1in} % set width of text portion

\begin{document}

\newcommand{\adress}{(adress:) XXXX}
\newcommand{\info}{phone / kaootao@gmail.com / kazukiotao.com}


% Center the name over the entire width of resume:
\moveleft.5\hoffset\centerline{\large\bf Kazki Otao}
% Draw a horizontal line the whole width of resume:
\moveleft\hoffset\vbox{\hrule width\resumewidth height 1pt}\smallskip
% address begins here
% Again, the address lines must be centered over entire width of resume:
\moveleft.5\hoffset\centerline{\adress}
\moveleft.5\hoffset\centerline{\info}

\begin{resume}

% \section{OBJECTIVE}
% A position in the field of computers with special
% interests in business applications programming,
% information processing, and management systems.

\section{Summary}

\begin{itemize}
  \item 10+ years of software development experience, including academic work.
  \item Professional experience in modern webfront, mobile apps, backend, machine learning, and data analysis (DS).
  \item Launched own service as Product Manager (PdM), and grow to 100,000 users.
  \item Certified as a ``Super Creator'' and ``Innovator'' by IPA, the Ministry of Economy, Trade, and Industry, Japan.
\end{itemize}

\section{PROFESSIONAL \\ EXPERIENCE}

{\sl Software Enginner \\ at Mercari Inc, Japan.} \hfill Apil. 2022 to Present. \\

\begin{itemize}  \itemsep -2pt
 \item User growth for an EC mobile app with 20 million MAUs (Swift).
\end{itemize}

{\sl CEO \\ at Telorain Inc, Japan.} \hfill Apil. 2020 to Present. \\

\begin{itemize}  \itemsep -2pt
  \item Launched mobile app service as PdM, and reached 100,000 users.
  \item Development of mobile app (Swift), research and development of video analysis algorithms (ML), development of backend (Docker, MySQL, gRPC, GCP).
  \item Development of live-streaming service (Kotlin, Swift, React, TypeScript).
\end{itemize}

{\sl External CTO \\ at Landscape Co.,Ltd., Japan.} \hfill Mar. 2020 to Jan. 2021 \\

\begin{itemize}  \itemsep -2pt
  \item Development of machine learning system to propose improvements for customer experience (ML, DS).
  \item Development of webfront for human resource management system (Anguler, TypeScript, Java).
  \item Development of crawler to collect customer information (Python).
\end{itemize}

{\sl Reserach Enginner / Sowftware Enginner \\ at Pixie Dust Technologies, Inc., Japan.} \hfill Sept. 2017 to Mar. 2020 \\

\begin{itemize}  \itemsep -2pt
  \item Research and development of retina projection display and retinal imaging camera. It inculdes development of image processing and computer vision technology, development of optical simulation software, development of optical design, patent applications, paper writing, and presentation at international conferences.
  \item Development of backend for management system (Python, Docker, AWS).
\end{itemize}

% {\sl Software Enginner \\ at Unirobot Corporation, Japan.} \hfill Dec. 2016 to Aug. 2017 \\

% \begin{itemize}  \itemsep -2pt
%   \item Development of UI/UX on the display on the home robot (Unity).
% \end{itemize}


\section{EDUCATION}

{\sl Master of Science in Informatics} (2022) \\
Graduate School of Library, Information and Media Studies, \\
University of Tsukuba, Japan.
% Content Engineering Laboratory (Adviser: Prof. Tetsuji Sato).

{\sl Bachelor of Science in Media Sciences and Engineering} (2019) \\
College of Media Arts, Science and Technology,
University of Tsukuba, Japan.
% Digital Nature Group (Adviser: Prof. Yoichi Ochiai).

{\sl Associate degree in Computer Science and Electronic Engineering} (2017) \\
Department of Computer Science and Electronic Engineering, \\
National Institute of Technology, Tokuyama College, Japan.
% Soft Computing Laboratory (Adviser: Prof. Takanori Koga).

\section{QUALIFICATIONS}

\begin{itemize}
  \item Innovatior, Mitou Advanced, IPA (2020)
  \item Super Creator, Mitou, IPA (2020)
  \item TOEIC 815 (2016)
  \item Applied Information Technology Engineer Examination, IPA (2013)
  \item Fundamental Information Technology Engineer Examination, IPA (2013)
\end{itemize}

\section{PUBLICATION}

{\bf Book} \\
1. Recent Developments and Prospective Applications of Aerial Display. CMC Publishing Co.,Ltd., 2018, 267p. (Written contribution of Part III, Chapter 9 )

{\bf International Conference (Selected)} \\
1. \underline{Kazuki Otao}, Yuta Itoh, Kazuki Takazawa, Hiroyuki Osone, and Yoichi Ochiai. 2018. Air Mounted Eyepiece: Optical See-Through HMD Design with Aerial Optical Functions. In Proceedings of the 9th Augmented Human International Conference (AH ’18). ACM.

2. Takanori Koga and \underline{Kazuki Otao}. 2018. An Interactive Fog Display to Express Adaptive Shape-Changing Flow. In the 25th International Display Workshops (IDW '18).

3. \underline{Kazuki Otao}, Yuta Itoh, Hiroyuki Osone, Kazuki Takazawa, Shunnosuke Kataoka, and Yoichi Ochiai. 2017. Light field blender: designing optics and rendering methods for see-through and aerial near-eye display. In SIGGRAPH Asia 2017 Technical Briefs (SA '17). ACM.

% \section{COMPUTER \\ SKILLS}

% {\sl Languages \& Software:}\\
% Python,
% C\#,
% Java,
% Swift,
% C/C++,
% Web (both frontend and backend),
% Git, LaTeX

% {\sl Framework \& Library:}\\
% Unity, Chainer

% {\sl Graphics:}\\
% Photoshop, After Effects, Premiere Pro, AviUtl, Blender, Illustrator

\end{resume}
\end{document}
